\subsection{Raggruppamento}
Il \textbf{\textcolor{purple}{raggruppamento}} è definito dall'espressione $_{\{A_i\}}\gamma_{\{f_i\}}(R)$,
dove gli $A_i$ sono gli attributi di $R$ mentre le $f_i$ sono funzioni di aggregazione.
Il \emph{raggruppamento} prima partiziona le ennuple di $R$ mettendo nello stesso gruppo
le ennuple con valori uguali solo sui campi degli $A_i$, successivamente si calcolano le
funzioni $f_i$ per ogni partizione. Infine per ogni partizione verrà restituita una solo ennupla
che avrà come attributi i valori degli $A_i$ e i valori delle funzioni $f_i$.

\subsection{Trasformazioni Algebriche}
Permettono di scegliere diversi ordini di \emph{join} e di anticipare
\emph{proiezioni} e \emph{selezioni} per lavorare su tabelle più piccole.

\begin{center}
    \textbf{\textcolor{purple}{Idempotenza delle Proiezioni}}
\end{center}
\begin{equation}
    \pi_{A}(\pi_{A,B}(R)) \equiv \pi_{A}(R)
\end{equation}

\begin{center}
    \textbf{\textcolor{purple}{Atomizzazione delle Selezioni}}
\end{center}
\begin{equation}
    \sigma_{Cond_{1}}(\sigma_{Cond_{2}}(R)) \equiv \sigma_{Cond_{1} \land Cond_{2}}(R)
\end{equation}

\begin{center}
    \textbf{\textcolor{purple}{Atomizzazione della Selezione rispetto al Join}} \\
    \emph{\textcolor{purple}{Pushing Selection Down}}
\end{center}
\begin{equation}
    \sigma_{Cond}(R \bowtie S) = R \bowtie \sigma_{Cond}(S)
\end{equation}
\begin{center}
    Se $Cond$ fà riferimento solo agli attributi di $S$.
\end{center}

\begin{center}
    \textbf{\textcolor{purple}{Anticipazione della Proiezione rispetto al Join}} \\
    \emph{\textcolor{purple}{Pushing Projections Down}}
\end{center}
\begin{equation}
    \pi_{X_1Y_2}(R \bowtie S) = R \bowtie \pi_{Y_2}(S)
\end{equation}
\begin{center}
    Se $R$ è definito su $X_1$ e $S$ su $X_2$, $Y_2 \subseteq X_2$ e gli attributi
    in $X_2 - Y_2$ non sono coinvolti nel join.
\end{center}

\begin{center}
    \textbf{\textcolor{purple}{Distributività della Selezione}}
\end{center}
\begin{equation}
    \sigma(R \cup S) = \sigma(R) \cup \sigma(S)
\end{equation}
\begin{equation}
    \sigma(R - S) = \sigma(R) - \sigma(S)
\end{equation}

\begin{center}
    \textbf{\textcolor{purple}{Distributività della Proiezione}}
\end{center}
\begin{equation}
    \pi_{X}(R \cup S) = \pi_{X}(R) \cup \pi_{X}(S)
\end{equation}
\paragraph{\textcolor{purple}{Nota Bene}} La \emph{proiezione} sulla \emph{differenza}
non gode della proprietà di \emph{distributività}:
\begin{equation}
    \pi_{X}(R - S) <> \pi_{X}(R) - \pi_{X}(S)
\end{equation}

\begin{center}
    \textbf{\textcolor{purple}{Inglobamento della Selezione nella Join}}
\end{center}
\begin{equation}
    \sigma_{Cond}(R \bowtie S) \equiv R \bowtie_{Cond} S
\end{equation}

\par\noindent\rule{\textwidth}{0.5pt}
\par

\begin{equation}
    \sigma_{Cond_{1} \land Cond_{2}}(R \times S) \equiv \sigma_{Cond_{1}}(R) \times \sigma_{Cond_{2}}(S)
\end{equation}

\begin{equation}
    \sigma_{Cond_{1} \lor Cond_{2}}(R) \equiv \sigma_{Cond_{1}}(R) \cup \sigma_{Cond_{2}}(R)
\end{equation}

\begin{equation}
    \sigma_{Cond_{1} \land Cond_{2}}(R \times S) \equiv \sigma_{Cond_{1}}(R) \cap \sigma_{Cond_{2}}(R)
\end{equation}

\begin{equation}
    \sigma_{Cond_{1} \land \neg Cond_{2}}(R \times S) \equiv \sigma_{Cond_{1}}(R) - \sigma_{Cond_{2}}(R)
\end{equation}

\begin{equation}
    R \times (S \times T) \equiv (R \times S) \times T
\end{equation}

\begin{equation}
    (R \times S) \equiv (S \times R)
\end{equation}

\begin{equation}
    \sigma_{Cond}(_X\gamma_F(R))\;\equiv\;_X\gamma_F(\sigma_{Cond}(R))
\end{equation}

\subsection{Operatori Non Insiemistici}

\paragraph{\textcolor{purple}{Proiezione MultiInsiemistica}}
$\pi^{b}_{A_i}(R)$, la $b$ sta ad indicare che le tuple duplicate non vanno eliminate.

\paragraph{\textcolor{purple}{Ordinamento}}
$\tau_{A_i}(R)$, ordina i valori degli attributi di ogni tupla seguendo l'ordine
degli $A_i$.