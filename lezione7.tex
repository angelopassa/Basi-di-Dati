\section{Algebra Relazionale}

\begin{definition}[Algebra Relazionale]
    Con \textbf{\textcolor{purple}{algebra relazionale}} si intende un
    insieme di \emph{operatori} su relazioni che danno come risultato altre relazioni.
    Non viene usato come linguaggio di interrogazione dei \emph{DBMS}, ma come rappresentazione
    interna delle interrogazioni.
\end{definition}

\subsection{Operatori Insiemistici}
Le relazioni vengono viste come degli \emph{\textcolor{purple}{insiemi}}.
Gli operatori di \emph{\textcolor{purple}{unione}}, \emph{\textcolor{purple}{differenza}}
ed \emph{\textcolor{purple}{intersezione}} sono applicabili solo a relazioni
definite sugli stessi \emph{attributi}.
\begin{itemize}
    \item \textbf{\textcolor{purple}{Unione}}: l'unione di due relazioni $R$ e $S$ definite
        sullo stesso insieme di attributi $X$ è indicata con $R \cup S$ ed è una relazione su $X$
        che contiene le tuple che appartengono a $R$ o a $S$ senza ripetizioni.
    \item \textbf{\textcolor{purple}{Differenza}}: la differenza è indicata con $R - S$ ed è una relazione sempre su
        $X$ che contiene tutte le tuple che appartengono ad $R$ ma non a $S$.
    \item \textbf{\textcolor{purple}{Ridenominazione}}: è un operatore \emph{\textcolor{purple}{monadico}}, ovvero su una sola relazione, che permette
        di modificare lo schema, ridenominando il nome di uno o più attributi, ma lasciando inalterate le istanze.
        \begin{center}
            $\rho_{\;Nuovo\;Nome\;Attributo\;\leftarrow\;Vecchio\;Nome\;Attributo}\;(Relazione)$
        \end{center}
    \item \textbf{\textcolor{purple}{Proiezione}}: è un altro operatore \emph{monadico} che produce un
        risultato su un sottoinsieme degli attributi della relazione e contiene le ennuple della
        relazione ristrette solo agli attributi del sottoinsieme.
        \begin{center}
            $\pi_{\;Lista\;Attributi}\;(Relazione)$
        \end{center}
        La cardinalità di una \emph{proiezione} è basata sul sottoinsieme $X$ degli
        attributi:
        \begin{itemize}
            \item Se $X$ è una \emph{superchiave} di $R$, allora $\pi_{X}(R)$ contiene esattamente
                tante ennuple quante ne contiene $R$.
            \item Altrimenti se $X$ non è \emph{superchiave}, potrebbero essere presenti dei valori che
                su quegli attributi sono ripetuti e che quindi vengono rappresentati una sola volta.
        \end{itemize}
    \item \textbf{\textcolor{purple}{Selezione}} o \textbf{\textcolor{purple}{Restrizione}}: è sempre un operatore
        \emph{monadico} che produce una relazione con lo stesso schema dell'operando, che contiene un sottoinsieme delle sue ennuple
        che soddisfano una data condizione.
        \begin{center}
            $\sigma_{\;Cond}\;(R)$
        \end{center}
        Con $Cond$ generata dalla seguente grammatica: \\ \\
        \verb+ Cond ::= Exp Theta Exp | Cond And Cond | Cond Or Cond | Not Cond + \\
        \verb+ Exp ::= Attributo | Costante | Exp Op Exp + \\
        \verb+ Theta ::= = | < | > | != | <= | >=+ \\
        \verb1 Op ::= + | - | * | StringConcat 1 \\ \\
        Le condizioni atomiche si riferiscono solo ai valori non nulli, per riferirsi anche
        ai valori nulli si usano forme di condizioni come \verb|IS NULL| e \verb|IS NOT NULL|.
    \item \textbf{\textcolor{purple}{Prodotto}}: operatore fra più relazioni che devono avere
        fra loro tutti i nomi degli attributi distinti, e che esegue il prodotto cartesiano fra gli
        insiemi di tuple.
    \item \textbf{\textcolor{purple}{Intersezione}}: l'intersezione è indicata con $R \cap S$ ed è una relazione definita
        sullo stesso insieme di attributi $X$ che contiene le tuple che appartengono sia ad $R$ che ad $S$.
        L'\emph{intersezione} è chiamato operatore \emph{\textcolor{purple}{derivato}}, dato che è possibile derivarlo usando altri
        operatori:
        \begin{center}
            $R \cap S \; = \; \{x \;|\; x \in R \land \exists \; y \in S \;t.c.\; x = y \}$
        \end{center}
        allora se per esempio $R$ ed $S$ sono definiti con $X = \{A, B\}$.
        \begin{center}
            $\pi_{A, B}(\sigma_{A=S.A\;AND\;B=S.B}(R \times \rho_{S}(S)))$
        \end{center}
        Partendo dall'interno, si ridenominano tutti gli attributi di $S$ ponendogli davanti
        il prefisso $S$, successivamente viene fatto il \emph{prodotto} con $R$ e tramite la
        \emph{selezione} selezioniamo solo le tuple che hanno gli attributi semanticamente uguali con
        gli stessi valori. Infine si fà una \emph{proiezione} per eliminare gli attributi ridondanti.
\end{itemize}

\subsubsection{Join}
Il \textbf{\textcolor{purple}{join}} o \textbf{\textcolor{purple}{giunzione}} permette di
correlare dati in relazioni diverse. Anch'esso è un operatore derivato in quanto si può ottenere per composizione
degli operatori visti precedentemente. Si indica con il simbolo $\bowtie$.

Il \textbf{\textcolor{purple}{join naturale}} produce un'unione degli attributi dei due operandi, combinando le ennuple
degli operandi che hanno valori uguali sugli attributi in comune. Ad esempio $R_1 \bowtie R_2$ è una relazione su $X_1 \cup X_2$ definita come segue:
\begin{center}
    $R_1 \bowtie R_2 = \{t \;su\; X_1 \cup X_2 \;|\; esistono \;t_1 \in R_1 \;e\; t_2 \in R_2 $ \\
    $\;con\; t[X_1] = t_1 \;e\; t[X_2] = t_2\}$
\end{center}

Un \emph{join} viene chiamato \emph{\textcolor{purple}{completo}} se ogni ennupla delle due relazioni
contribuisce al risultato. Al contrario si parla di \emph{join} \emph{\textcolor{purple}{non completo}}.
Generalmente possiamo definire un limite inferiore e superiore alla cardinalità di una join:
\begin{center}
    $0 \leq |R_1 \bowtie R_2| \leq |R_1| \times |R_2|$
\end{center}
Nel caso il join è \emph{completo} possiamo restringere il limite inferiore
dicendo che $|R_1 \bowtie R_2| \geq \max\{|R_1|, |R_2|\}$.
\\\\
Nel caso in cui il join coinvolge come attributo comune una \emph{chiave} di $R_2$, allora:
\begin{center}
    $0 \leq |R_1 \bowtie R_2| \leq |R_1|$
\end{center}

Se oltre a coinvolgere una \emph{chiave} di $R_2$, vi è un vincolo di \emph{integrità referenziale} tra un
attributo di $R_1$ e la chiave di $R_2$ allora la cardinalità della join è esattamente $|R_1|$.
\\\\
Un \textbf{\textcolor{purple}{join esterno}} estende con valori nulli le ennuple che verrebbero
tagliate fuori da un \emph{\textcolor{purple}{join naturale}}.
Esiste in tre versioni:
\begin{itemize}
    \item \textcolor{purple}{Join esterno sinistro} $R \overset{\leftarrow}{\bowtie} S$: mantiene tutte le ennuple del primo
        operando, estendendole con valori nulli se necessario.
    \item \textcolor{purple}{Join esterno destro} $R \overset{\rightarrow}{\bowtie} S$: mantiene tutte le ennuple del secondo
    operando, estendendole con valori nulli se necessario.
    \item \textcolor{purple}{Join esterno completo} $R \overset{\leftrightarrow}{\bowtie} S$: mantiene tutte le ennuple di entrambi
        gli operandi, estendendole con valori nulli se necessario.
\end{itemize}

\paragraph{\textcolor{purple}{Nota Bene}} Il \emph{prodotto cartesiano} si può
vedere come un \emph{join naturale} su relazioni che non hanno attributi in comune.
\\ \\
Dato che il \emph{\textcolor{purple}{prodotto cartesiano}} non ha senso se non
lo facciamo seguire da una selezione (che permette di specificare la condizione sui campi
che semanticamente rappresentano lo stesso attributo), si usa l'operazione di
\textbf{\textcolor{purple}{theta-join}} definito come segue:
\begin{center}
    $R_1 \bowtie_{\;Condizione} R_2$
\end{center}
che è equivalente a:
\begin{center}
    $\sigma_{\;Condizione}\;(R_1 \bowtie R_2)$
\end{center}
\textbf{N.B.}: dato che $R_1$ ed $R_2$ non hanno attributi comuni, in
questo caso $\bowtie \;= \times$.

Se nel \emph{theta-join} l'operatore di confronto nella $Condizione$ è sempre
l'uguaglianza, allora si parla di \textbf{\textcolor{purple}{equi-join}}.

\paragraph{\textcolor{purple}{Self Join}} Supponiamo di avere questa relazione \verb|Genitori|:
\begin{center}
    \begin{tabular}{|c|c|}
        \hline
        \rowcolor{purple}
        \textcolor{white}{Genitore} & \textcolor{white}{Figlio} \\
        \hline
        Luca & Anna \\
        \hline
        Maria & Anna \\
        \hline
        Giorgio & Luca \\
        \hline
        Silvia & Maria \\
        \hline
        Enzo & Maria \\
        \hline
    \end{tabular}
\end{center}

e di voler ottenere una relazione \verb|Nonno-Nipoti|. Per far ciò occorre riutilizzare la stessa
tabella ma facendo $Genitori \bowtie Genitori$ si otterrebbe di nuovo la tabella \verb|Genitori|.
In questo caso occorre ridenominare la tabella due volte:

\begin{table}[H]
    \parbox{.55\linewidth}{
    \centering
    \begin{tabular}{|c|c|}
        \hline
        \rowcolor{purple}
        \textcolor{white}{Nonno} & \textcolor{white}{Genitore} \\
        \hline
        Luca & Anna \\
        \hline
        Maria & Anna \\
        \hline
        Giorgio & \textcolor{blue}{Luca} \\
        \hline
        Silvia & \textcolor{red}{Maria} \\
        \hline
        Enzo & \textcolor{red}{Maria} \\
        \hline
    \end{tabular}
    \caption{$\rho_{\;Nonno,\;Genitore \leftarrow Genitore,\;Figlio}(Genitore)$}
    }
    \hfill
    \parbox{.45\linewidth}{
    \centering
    \begin{tabular}{|c|c|}
        \hline
        \rowcolor{purple}
        \textcolor{white}{Genitore} & \textcolor{white}{Nipote} \\
        \hline
        \textcolor{blue}{Luca} & Anna \\
        \hline
        \textcolor{red}{Maria} & Anna \\
        \hline
        Giorgio & Luca \\
        \hline
        Silvia & Maria \\
        \hline
        Enzo & Maria \\
        \hline
    \end{tabular}
    \caption{$\rho_{\;Nipote \leftarrow Figlio}(Genitore)$}
    }
\end{table}
Infine, effettuando una \emph{join} seguita da una \emph{proiezione} otteniamo:
\begin{center}
    \begin{tabular}{|c|c|}
        \hline
        \rowcolor{purple}
        \textcolor{white}{Nonno} & \textcolor{white}{Nipote} \\
        \hline
        Giorgio & Anna \\
        \hline
        Silvia & Anna \\
        \hline
        Enzo & Anna \\
        \hline
    \end{tabular}
\end{center}

\begin{center}
    $\pi_{\;Nonno,\;Nipote}(\rho_{\;Nonno,\;Genitore \leftarrow Genitore,\;Figlio}(Genitore) \bowtie \rho_{\;Nipote \leftarrow Figlio}(Genitore))$
\end{center}