\section{SQL}

\subsection{Query Language}
Le interrogazioni al database vengono fatte mediante il comando
\verb|SELECT|, nel quale occorre specificare la lista degli attributi interessati.
Il comando presenta anche due clausole, \verb|FROM| che indica in quali tabelle
sono contenuti i dati, e \verb|WHERE| (\emph{opzionale}) per esprimere le condizioni che i dati
devono soddisfare.
\begin{lstlisting}[
    language=SQL,
    showspaces=false,
    basicstyle=\ttfamily,
    numbers=left,
    numberstyle=\tiny,
    commentstyle=\color{gray}
 ]
SELECT Lista Attributi
FROM Lista Tabelle
[WHERE Condizione]
\end{lstlisting}

Specificare la \emph{Lista Attributi}, anche chiamata \textbf{\textcolor{purple}{target list}}
rappresenta l'operazione di \emph{\textcolor{purple}{proiezione}}.
La clausola \verb|WHERE|, invece permette di implementare la \emph{\textcolor{purple}{selezione}}.

Il \textbf{\textcolor{purple}{theta-join}}, invece è implementato indicando nella clausola
\verb|FROM| le due tabelle, e nel \verb|WHERE| la condizione che stabilisce quali righe accoppiare.

\begin{lstlisting}[
    language=SQL,
    showspaces=false,
    basicstyle=\ttfamily,
    numbers=left,
    numberstyle=\tiny,
    commentstyle=\color{gray}
 ]
SELECT *
FROM Studenti, Esami
WHERE Studenti.Matricola = Esami.Studenti
\end{lstlisting}

Nel caso in cui la clausola \verb|WHERE| non è presente, la query sopra indicata esegue il prodotto cartesiano
anche chiamato \emph{\textcolor{purple}{cross-join}}.

\paragraph{\textcolor{purple}{DISTINCT}} \emph{SQL} è un linguaggio che
lavora su \emph{multiinsiemi}, quindi di \emph{default}, a seguito di una \emph{query},
potrebbero essere presenti righe identiche. Per evitare ciò basta specificare questo facendo seguire il comando
\verb|SELECT| dalla clausola \verb|DISTINCT|.

\paragraph{\textcolor{purple}{Ridenominazione}} Per ridenomiare un attributo, basta farlo seguire da
\verb|AS Nuovo Nome Attributo|.
Mentre per ridenomiare una tabella, basta far seguire il nome originale nella clausola \verb|FROM| dal nuovo nome.

\paragraph{\textcolor{purple}{JOIN}} L'operazione di \emph{\textcolor{purple}{Join}}, oltre che essere implementato
come visto prima tramite la combinazione di \verb|FROM| e \verb|WHERE|, è possibile codificarla in modo esplicito, ecco alcuni esempi:
\begin{itemize}
    \item \verb|Studenti s JOIN Esami e ON s.Matricola = e.Matricola|
    \item \verb|Studenti s JOIN Esami e USING Matricola|
    \item \verb|Studenti s NATURAL JOIN Esami|
    \item \verb|Studenti s LEFT [OUTER] JOIN Esami e| \\
        \verb|ON s.Matricola = e.Matricola|
\end{itemize}

\paragraph{\textcolor{purple}{Self-Join}} Per implementare il \emph{self-join} occorre prima ridenominare
la tabella due volte:
\begin{lstlisting}[
    language=SQL,
    showspaces=false,
    basicstyle=\ttfamily,
    numbers=left,
    numberstyle=\tiny,
    commentstyle=\color{gray}
 ]
SELECT T1.*, T2.*
FROM Tabella T1, Tabella T2
WHERE T1.codice = T2.codice
\end{lstlisting}

\paragraph{\textcolor{purple}{LIKE}} Questo operatore è usato per effettuare dei \emph{pattern matching} con una stringa
tramite l'uso di due \emph{\textcolor{purple}{wildcard}}:
\begin{itemize}
    \item \textbf{\textcolor{purple}{\%}} : che denota la presenza di $0$ o più caratteri.
    \item \textbf{\textcolor{purple}{\_}} : presenza di esattamente $1$ carattere.
\end{itemize}

Nel caso si voglione usare le \emph{wildcards} nel \emph{patter matching}, si imposta un carattere di \emph{\textcolor{purple}{escape}}:
\begin{lstlisting}[
    language=SQL,
    showspaces=false,
    basicstyle=\ttfamily,
    numbers=left,
    numberstyle=\tiny,
    commentstyle=\color{gray}
 ]
SELECT *
FROM Modelli
WHERE nome_modello LIKE 'C#_F%' ESCAPE '#'
\end{lstlisting}

\paragraph{\textcolor{purple}{NULL}} Quando si confronta un attributo
con \verb|NULL| è sempre consigliabile farlo tramite gli operatori \verb|IS [NOT] NULL|, dato che
il confronto tramite l'uguaglianza, nel caso in cui il valore dell'attributo sia \verb|NULL|, restituisce \emph{valore sconosciuto},
ovvero \verb|NULL|, e non un valore \emph{booleano}.
Nel calcolo di espressioni, inoltre, se si incontra
un valore \verb|NULL|, viene sempre restituito \emph{valore sconosciuto}.

\paragraph{\textcolor{purple}{ORDER BY}} Permette di dare un ordinamento
del risultato della \verb|SELECT|, basandosi sui valori di uno o più attributi e opzionalmente
indicando in maniera esplicita se l'ordine deve essere crescente (\verb|ASC|, predefinito) o decrescente
(\verb|DESC|).

\subsubsection{Operatori Aggregati}

Nella \emph{target list} possiamo avere anche espressioni che calcolano
valori a partire da un insieme di ennuple, e che restituiscono un singolo valore
scalare. \emph{SQL} prevede 5 operatori aggregati:
\begin{itemize}
    \item \textbf{\textcolor{purple}{Count}}: restituisce il numero di righe del risultato della \emph{query}. Mediante
        la specifica \verb|(*)| si contano tutte le righe selezionate, con \verb|ALL| (specifica di \emph{default}) si contano tutte le righe selezionate che non sono \verb|NULL|,
        mentre con \verb|DISTINCT| si contano tutte le righe non nulle con valori distinti.
    \item \textbf{\textcolor{purple}{Max \& Min}}: calcolano rispettivamente il massimo e il minimo degli elementi presenti
        in un colonna dello schema.
    \item \textbf{\textcolor{purple}{Avg}}: calcola la media dei valori non nulli su una colonna, si possono utilizzare le specifiche \verb|ALL| e \verb|DISTINCT|.
    \item \textbf{\textcolor{purple}{Sum}}: somma tutti i valori non nulli su una colonna, anche qui si trovano le specifiche \verb|ALL| e \verb|DISTINCT|.
\end{itemize}

\paragraph{\textcolor{purple}{GROUP BY}} Se si vogliono applicare gli operatori aggregati non sull'intera
tabella ma raggruppando i valori per un sottoinsieme di attributi si utilizza la clausola \verb|GROUP BY|, che specifica
gli attributi su cui fare i raggruppamenti. Le funzioni di aggregazione saranno applicate su ogni gruppo.

\paragraph{\textcolor{purple}{HAVING}} Tramite questa clausola si possono applicare condizioni sui valori aggregati per ogni gruppo.

\paragraph{\textcolor{purple}{SottoSelect} o \textcolor{purple}{SubQuery}} È possibile innestare una \verb|SELECT| in un'altra, tramite le parole chiave
\verb|ANY|, \verb|ALL|, \verb|[NOT] IN|, \verb|[NOT] EXISTS|, più tutte gli altri operatori della clausola \verb|WHERE|, seguita dalla \emph{sottoselect}.
\begin{lstlisting}[
    language=SQL,
    showspaces=false,
    basicstyle=\ttfamily,
    numbers=left,
    numberstyle=\tiny,
    commentstyle=\color{gray}
 ]
SELECT *
FROM Studenti
WHERE voto > (SELECT AVG(voto)
              FROM Studenti)
\end{lstlisting}

Oppure è possibile fare operazioni insiemistiche tra i risultati di due \verb|SELECT| tramite gli operatori
\verb|UNION|, \verb|INTERSECT|, \verb|EXCEPT|.

Ci sono tre tipologie di \emph{subquery}:
\begin{itemize}
    \item Subquery \textcolor{purple}{Scalari}: una \verb|SELECT| che restituisce un solo valore.
    \item Subquery di \textcolor{purple}{Colonna}: restituisce una colonna.
    \item Subquery di \textcolor{purple}{Tabella}: restituisce una tabella.
\end{itemize}

Nelle query nidificate le \emph{regole di visibilità} sono simili a quelle dei linguaggi
di programmazione, ovvero all'esterno non è possibile riferirsi a variabili definite in \emph{query} più interne o nello stesso livello, viceversa sì.

\paragraph{NOTA} Nelle \emph{subquery} non è possibile utilizzare le clausole \verb|HAVING| e \verb|GROUP BY|.

\paragraph{\textcolor{purple}{Quantificazione}} È importante ricordarsi due \emph{"trasformazioni"} molto utili per la quantificazione:
\begin{itemize}
    \item L'\textcolor{purple}{universale negata} è uguale all'\textcolor{purple}{esistenziale}: \emph{non tutti i voti sono $\leq$ 24} vale a dire \emph{almeno un voto $>$ 24}.
    \item L'\textcolor{purple}{esistenziale negata} è uguale all'\textcolor{purple}{universale}: \emph{non esiste un voto diverso da 30} vale a dire \emph{tutti i voti sono uguali a 30}.
\end{itemize}

Quando si utilizza nel \verb|WHERE| un operatore scalare seguito da \verb|ANY| più una \emph{subquery}, il predicato sarà vero solo se almeno
uno dei valori restituiti dalla \emph{subquery} lo soddisfa. Mentre se si usa \verb|ALL|, tutti i valori restituiti devono soddisfarlo.

La keyword \verb|EXISTS| invece rappresenta un quantificatore esistenziale, ciò rende vero il predicato se la \emph{subquery}
che lo segue restituisce almeno una tupla.

È bene notare che utilizzare \verb|IN| equivale alla forma \verb|=ANY|. \\

Gli operatori definiti di seguito lavorano su tabelle omogenee, quindi definite sullo stesso
dominio.

\paragraph{\textcolor{purple}{UNION}} L'operatore \verb|UNION| prende come operandi i risultati di
due \verb|SELECT| e restituisce una terza tabella che contiene l'unione delle righe senza duplicati. Se è scecificata l'opzione
\verb|ALL| allora si mantengono anche le righe doppie.
Nella tabella risultante i nomi degli attributi vengono presi da quelli del primo operando.

\paragraph{\textcolor{purple}{EXCEPT}} Questo operatore rappresenta la differenza insiemistica. Quindi si mantengono
tutte le tuple della prima tabella che non sono presenti anche nella seconda. Anche qui si può utilizzare l'opzione \verb|ALL|.

\paragraph{\textcolor{purple}{INTERSECT}} Questo operatore realizza l'intersezione dell'algebra relazionale, quindi restituisce
solo le righe comuni alle due tabelle che restituiscono le due \verb|SELECT|. L'opzione \verb|ALL| è disponibile anche per questo operatore.

\subsection{Data Manipulation Language}

Il \emph{DML} è il linguaggio di \emph{SQL} per inserire, modificare e cancellare dati nel database e per interrogarlo.

\paragraph{\textcolor{purple}{INSERT}} Per inserire dati nel database si utilizza la seguente sintassi:
\begin{lstlisting}[
    language=SQL,
    showspaces=false,
    basicstyle=\ttfamily,
    numbers=left,
    numberstyle=\tiny,
    commentstyle=\color{gray}
 ]
INSERT INTO Nome_Tabella
[(Lista_Attributi)] VALUES (Lista_Valori)
\end{lstlisting}

Le \verb|[]| indicano che specificare la lista degli attributi è opzionale, in caso non è specificata si è obbligati ad assegnare valori a tutti gli attributi
nell'ordine opportuno. Se a un attributo non viene assegnato alcun valore, viene assegnato automaticamente \verb|NULL|, a meno che sia
stato specificato in precedenza un valore di \emph{default}. Se invece, \verb|NULL| non è permesso per gli attributi mancanti, l'inserimento fallisce.

È possibile effettuare un inserimento prendendo dati da un'altra tabella in questo modo:
\begin{lstlisting}[
    language=SQL,
    showspaces=false,
    basicstyle=\ttfamily,
    numbers=left,
    numberstyle=\tiny,
    commentstyle=\color{gray}
 ]
INSERT INTO Tabella_1(Attributo_1, ..., Attributo_n)
SELECT Attributo_1, Attributo_n
FROM Tabella_2
\end{lstlisting}

\paragraph{\textcolor{purple}{DELETE}} Per cancellare tuple da una tabella si utilizza la seguente sintassi:
\begin{lstlisting}[
    language=SQL,
    showspaces=false,
    basicstyle=\ttfamily,
    numbers=left,
    numberstyle=\tiny,
    commentstyle=\color{gray}
 ]
DELETE FROM Tabella
[WHERE Condizione]
\end{lstlisting}
Specificando la clausola \verb|WHERE| si vanno ad eliminare solo le tuple che rispettono quella \emph{Condizione}.
Nella clausola \verb|WHERE| è possibile utilizzare una \emph{subquery}.

\paragraph{\textcolor{purple}{UPDATE}} Per aggiornare gli attributi di alcune tuple utilizzando il comando \verb|UPDATE|:
\begin{lstlisting}[
    language=SQL,
    showspaces=false,
    basicstyle=\ttfamily,
    numbers=left,
    numberstyle=\tiny,
    commentstyle=\color{gray}
 ]
UPDATE Tabella
SET Nome_Attributo = Espressione
WHERE Condizione
\end{lstlisting}

\subsection{Data Definition Language}

Il linguaggio \emph{DDL} permette la creazione, modifica, cancellazione di tabelle,
domini e degli altri oggetti del database. Il fine è quello di definire lo \emph{schema logico}
del database.

Per la creazione di tabelle si utilizza il comando \verb|CREATE TABLE|:
\begin{lstlisting}[
    language=SQL,
    showspaces=false,
    basicstyle=\ttfamily,
    numbers=left,
    numberstyle=\tiny,
    commentstyle=\color{gray}
 ]
CREATE TABLE Nome_Tabella
(
  Nome_Colonna_1, Tipo_1, Clausola_Default_1, Vincolo_1,
  ...
  Nome_Colonna_n, Tipo_n, Clausola_Default_n, Vincolo_n,
  Vincoli_Tabella
)
\end{lstlisting}

Questo comando definisce uno schema di relazione e ne crea un'istanza vuota.
Successivamente a questo comando, la tabella è pronta per l'inserimento.

\paragraph{\textcolor{purple}{DESCRIBE}} Il comando \verb|DESCRIBE| permette di ottenere
lo schema di una tabella dopo che è stata creata.

Un attributo può avere uno dei seguenti tipi:
\begin{itemize}
    \item \textcolor{purple}{CHAR(n)}: stringhe di lunghezza esattamente $n$.
    \item \textcolor{purple}{VARCHAR(n)}: stringhe di lunghezza al più $n$.
    \item \textcolor{purple}{INTEGER}: interi di dimensione uguale alla lunghezza della \emph{word}
        di memoria dell'elaboratore.
    \item \textcolor{purple}{REAL}: numeri reali di dimensione uguale alla lunghezza della \emph{word}
    di memoria dell'elaboratore.
    \item \textcolor{purple}{NUMBER(p, s)}: numeri di $p$ cifre di cui $s$ decimali.
    \item \textcolor{purple}{FLOAT(p)}: numeri binari in virgola mobile con almeno $p$ cifre significative.
    \item \textcolor{purple}{DATE}.
\end{itemize}

\paragraph{\textcolor{purple}{ALTER TABLE}} Con questo comando è possibile modificare lo schema
di una tabella:
\begin{itemize}
    \item \textcolor{purple}{ADD [Nome Attributo]} per aggiungere una nuova colonna. Se non ci sono
        altre specifiche l'attributo viene aggiunto come ultima colonna, altrimenti con \verb|FIRST| si aggiunge l'attributo
        come prima colonna, e con \verb|AFTER Nome Attributo| si aggiunge subito dopo quello indicato. 
    \item \textcolor{purple}{DROP [Nome Attributo]} per eliminare una colonna. Se un'altra tabella presenta un vincolo di integrità
        referenziale con l'attributo da eliminare, utilizzando la specifica \verb|RESTRICT|, il comando fallisce; altrimenti
        utilizzando \verb|CASCADE| la colonna viene eliminata, ed anche tutte le dipendenze logiche (vincolo di \emph{foreign key}) che altre tabelle
        hanno con questo attributo. 
    \item \textcolor{purple}{MODIFY} per modificare una colonna.
    \item \textcolor{purple}{SET DEFAULT} per assegnare valori di default agli attributi.
    \item \textcolor{purple}{DROP DEFAULT} per eliminare l'assegnazione di valori di default.
         \begin{lstlisting}[
            language=SQL,
            showspaces=false,
            basicstyle=\ttfamily,
            numbers=left,
            numberstyle=\tiny,
            commentstyle=\color{gray}
        ]
        ALTER TABLE Tabella
        ALTER [COLUMN] Colonna
        SET/DROP DEFAULT
        \end{lstlisting}
    \item \textcolor{purple}{ADD CONSTRAINT} per aggiungere vincoli di tabella.
    \item \textcolor{purple}{DROP CONSTRAINT} per eliminare vincoli di tabella. Anche qui sono presenti
        le specifiche \verb|RESTRICT| e \verb|CASCADE|. Infatti con \verb|RESTRICT| non è permesso eliminare
        i vincoli di unicità e di chiave primaria su una colonna se esistono vincoli di chiave esterna che si riferiscono
        alla stessa colonna.
\end{itemize}

Se l'attributo da definire presenta il vincolo \verb|NOT NULL|, occorre prima aggiungerlo
senza il vincolo, poi per ogni tupla della tabella si aggiunge un valore per quell'attributo, ed
infine si può modificare la tabella impostando quell'attributo come \verb|NOT NULL|.

\paragraph{\textcolor{purple}{DROP TABLE}} Per eliminare una tabella. Utilizzando anche qui \verb|RESTRICT|, l'eliminazione
è impedita se la tabella è utilizzata nella definizione di altri oggetti. Con \verb|CASCADE|, invece, vengono eliminate anche tutte
le dipendenze che gli altri oggetti hanno con questa tabella.

\subsubsection{Vincoli}

I vincoli di \emph{\textcolor{purple}{integrità}} consentono di limitare i valori ammissibile
per un'attributo. Quelli \emph{intrarelazionali} (che non si riferiscono ad altre tabelle) sono:
\begin{itemize}
    \item \textcolor{purple}{NOT NULL}.
    \item \textcolor{purple}{UNIQUE} serve per definire una chiave, e può essere utilizzato sia nella definizione
        di un attributo, se ne coinvolge solo uno, o per un insieme di attributi, in questo caso viene definito nei vincoli
        di tabella e si indica la lista di attributi che formano la chiave.
    \item \textcolor{purple}{PRIMARY KEY} anche qui può essere definito nella definizione di attributo, se ne coinvolge solo una, o nei vincoli di tabella
        se la chiave primaria è formata da più attributi. Questo vincolo implica sia il vincolo \textcolor{purple}{UNIQUE} sull'insieme di attributi,
        che il vincolo \textcolor{purple}{NOT NULL} per ogni attributo della chiave primaria.
    \item \textcolor{purple}{CHECK} è un'espressione booleana, e richiede che un attributo o un insieme di attributi soddisfino una condizione per ogni tupla della tabella.
\end{itemize}

I vincoli \emph{interrelazionali}, invece, si impostano per mettere in relazione attributi di
tabelle diverse. Queste relazioni sono importanti per eliminare il più possibile la ridondanza dei dati.
Per impostare un vincolo di integrità referenziale su un singolo attributo, nella sua descrizione si indica con \verb|REFERENCES Tabella_Riferita(Attributo_Riferito)|.
Mentre se il vincolo coinvolge più attributi, anche qui bisogna definirlo nei vincoli di tabella con \verb|FOREIGN KEY (Lista_Attributi_Referenti)|
\verb|REFERENCES Tabella_Riferita(Lista_Attributi_Riferiti)|.

È possibile definire delle politiche di reazione alla violazione quando uno di questi vincoli viene violato medianti i comandi \verb|ON DELETE| e
\verb|ON UPDATE|. Le azioni per la \verb|ON DELETE| possono essere le seguenti:
\begin{itemize}
    \item \textcolor{purple}{NO ACTION}: questa è l'azione di \textbf{default} e blocca la cancellazione
        delle righe nella tabella riferita.
    \item \textcolor{purple}{CASCADE}: cancella tutte le righe dipendenti se si riferiscono ad una riga che viene
        eliminata dalla tabella riferita.
    \item \textcolor{purple}{SET NULL}: assegna \verb|NULL| ai valori delle colonne con il vincolo di \emph{foreign key},
        se dipendono da una riga cancellata dalla tabella riferita.
    \item \textcolor{purple}{SET DEFAULT}: assegna il valore di \emph{default} ai valori delle colonne con il vincolo di \emph{foreign key},
        se dipendono da una riga cancellata dalla tabella riferita.
\end{itemize}

Le azioni per \verb|ON UPDATE| sono le stesse ma hanno una semantica diversa:
\begin{itemize}
    \item \textcolor{purple}{NO ACTION}: rifiuta gli aggiornamenti che violano l'integrità referenziale.
    \item \textcolor{purple}{CASCADE}: le righe della tabella referente vengono impostate con i nuovi valori aggiornati della tabella riferita.
    \item \textcolor{purple}{SET NULL}: i valori della tabella referente sono impostati a \verb|NULL|.
    \item \textcolor{purple}{SET DEFAULT}: i valori della tabella referente sono impostati al valore di \emph{default}.
\end{itemize}

\subsubsection{Viste}

Le \textbf{\textcolor{purple}{viste}} sono delle tabelle virtuali i cui dati sono
riaggregazioni dei dati contenuti nelle tabelle fisiche del database.
Contengono gli stessi dati, ma forniscono una visione diversa e dinamicamente aggiornata.

Permettono di proteggere i database, dato che consentono di limitare l'accesso ai dati, di rappresentare
i dati in modo più significativo per l'utente, ma non è possibile utilizzare alcuni operatori come 
\verb|UNION|, \verb|INTERSECT| ed \verb|EXCEPT|, e neanche la clausola \verb|ORDER BY|.

Il vantaggio principale è che garantiscono l'\textbf{\textcolor{purple}{indipendenza logica}} delle
applicazioni rispetto alla struttura del database, cioè è possibile operare modifiche agli schemi
nel database senza dover apportare modifiche anche nelle applicazioni che lo usano.

Il comando da utilizzare è il seguente:
\begin{lstlisting}[
    language=SQL,
    showspaces=false,
    basicstyle=\ttfamily,
    numbers=left,
    numberstyle=\tiny,
    commentstyle=\color{gray}
]
CREATE VIEW Nome_Vista [(Lista_Attributi)] AS Select
[with [local | cascaded] check option]
\end{lstlisting}
Nella lista attributi si trovano i nomi assegnati alle colonne che corrispondono
ordinatamente alle colonne selezionate dalla \verb|SELECT|, se sono omesse si utilizzano gli stessi nomi
usati nella \verb|SELECT|.

\paragraph{Nota Bene} Solo il \emph{contenuto} di una vista è
\emph{dinamico}, la sua \emph{struttura} non lo è. Se in seguito
alla creazione della vista, alla tabella (o le tabelle) utilizzata nella \verb|SELECT|
viene aggiunta una colonna, questa non sarà presente nella vista.

\paragraph{\textcolor{purple}{Vista di Gruppo}} È una vista in cui una delle colonna è una
funzione di aggregazione, in questo caso è obbligatorio assegnare un nome alla colonna corrispondente
alla funzione. Sono chiamate \emph{\textcolor{purple}{viste di gruppo}} anche quelle in cui
la \verb|SELECT| lavora su altre viste di gruppo.

\paragraph{\textcolor{purple}{DROP VIEW}} Per eliminare una vista si usa il comando:
\begin{lstlisting}[
    language=SQL,
    showspaces=false,
    basicstyle=\ttfamily,
    numbers=left,
    numberstyle=\tiny,
    commentstyle=\color{gray}
]
DROP VIEW Nome_Vista {RESTRICT / CASCADE}
\end{lstlisting}
Con \verb|RESTRICT| la vista viene eliminata solo se non è riferita nella definizione di altri oggetti.
Con \verb|CASCADE|, invece vengono eliminate anche tutte le dipendenze da tale vista.

Le tabelle delle viste si possono interrogare come le altre, ma in generale non si possono
modificare, a meno che ci sia una corrispondenza biunivoca fra le righe della tabella e quelle della vista,
ovvero se nella \verb|SELECT| non sia presente la \verb|DISTINCT| e ci siano solo attributi, nella \verb|FROM|
ci deve essere una sola tabella modificabile, il \verb|WHERE| deve essere senza \emph{subquery} e non ci devono essere
\verb|GROUP BY| ed \verb|HAVING|.

L'opzione \verb|with check option| assicura che le operazioni di inserimento e modifica
dei dati sulla vista soddisfino la clausola \verb|WHERE| nella \verb|SELECT| utilizzata nella
creazione della vista.

Nel caso in cui si utilizza la specifica \verb|CASCADED| (di \emph{default}), se la vista è definita
in termini di altre viste, allora il DBMS controlla che la nuova tupla soddisfi sia la clausola \verb|WHERE| della vista
principale che di quelle di cui è dipendente, indipendemente che quelle viste sono state definiti con l'opzione \verb|with check option|.
Invece con la specifica \verb|LOCAL| si verifica solo che soddisfi le clausole delle viste di cui la principale è dipendente con l'opzione
\verb|with check option| abilitata.