\section{Progettazione Logica}

\subsection{Relazioni}

\begin{definition}[Relazione Matematica]
    Una \emph{relazione matematica} è un insieme di \emph{n}-uple ordinate
    $(d_1, \dots, d_n)$ tali che $d_1 \in D_1, \dots, d_n \in D_n$.

    Essendo un insieme, non c'è ordinamento fra le \emph{n}-uple, che inoltre devono essere
    tutte distinte. Però l'ordinamento all'interno della \emph{n}-upla conta, infatti l'\emph{i}-esimo
    valore deve provenire dall'\emph{i}-esimo dominio. Per questo si dice che la struttura della relazione
    è \emph{\textcolor{purple}{posizionale}}.
\end{definition}

\begin{definition}[Attributo]
    A ciascun dominio della relazione si associa un nome, chiamata \textbf{\textcolor{purple}{attributo}}.
\end{definition}

\begin{definition}[Tupla]
    Una \textbf{\textcolor{purple}{tupla}} su un insieme di \emph{attributi} chiamato $X$, è una funzione
    $t$ che associa a ciascun \emph{attributo} un valore del suo \emph{dominio}.

    Una \emph{\textcolor{purple}{relazione su $X$}} è un insieme di \emph{tuple}
    su $X$.
\end{definition}

\subsection{Tabelle}
Una \textbf{\textcolor{purple}{tabella}} rappresenta una relazione se:
\begin{itemize}
    \item I valori di ogni \emph{colonna} sono fra loro \emph{omogenei}.
    \item Le \emph{righe} sono tutte \emph{diverse} fra loro.
    \item Le \emph{intestazioni} delle colonne sono tutte \emph{diverse} fra loro.
\end{itemize}

In una \emph{tabella} l'ordinamento tra le righe e le colonne è irrilevante.

In ogni tabella è possibile distinguere due parti:
\begin{itemize}
    \item Lo \textbf{\textcolor{purple}{schema}} è rappresentato dalle intestazioni della tabella, che sono invarianti nel tempo e descrivono la
        struttura della tabella (\textbf{\textcolor{purple}{aspetto intensionale}}).
    \item L'\textbf{\textcolor{purple}{istanza}} sono i valori attuali presenti nella tabella, che
        possono cambiare nel tempo (\textbf{\textcolor{purple}{aspetto estensionale}}).
\end{itemize}

\begin{definition}[Tipo Ennupla]
    Un \textcolor{purple}{tipo ennupla} chiamato $T$ è un insieme finito di coppie (\emph{Attributo}, \emph{Tipo Elementare}).
\end{definition}

\begin{definition}[Schema di Relazione]
    Se $T$ è un \emph{tipo ennupla}, allora $R(T)$ è lo \textcolor{purple}{schema della relazione $R$}.
\end{definition}

\begin{definition}[Schema di Database]
    Lo \textcolor{purple}{schema di un database} è un insieme di \emph{schemi di relazioni} $R_i(T_i)$.
\end{definition}

\begin{definition}[Istanza di Relazione]
    Un'\textcolor{purple}{istanza di relazione}, anche chiamata \textbf{\textcolor{purple}{relazione}} su uno
    schema $R(T)$ è l'insieme $r$ di tuple di tipo $T$.
\end{definition}

\begin{definition}[Istanza di Database]
    Un'\textcolor{purple}{istanza di database} su uno schema $R=\{R_1(T_1), \dots, R_n(T_n)\}$
    è l'insieme delle \emph{relazioni} $r=\{r_1, \dots, r_n\}$, dove $r_i$ è un'\emph{istanza di relazione} su $R_i(T_i)$.
\end{definition}

\paragraph{Valore Nullo} Si indica con \verb|NULL|, e indica l'assenza di un valore del dominio.

\subsection{Vincoli di Integrità}
Un \textbf{\textcolor{purple}{vincolo d'integrità}} è una proprietà che deve
essere soddisfatta da ogni singola \emph{istanza} della relazione, in modo tale che rappresenti informazioni
corrette per l'applicazione.

Il vincolo viene espresso mediante un predicato, che associa ad ogni \emph{istanza} il valore \emph{\textcolor{Green}{vero}}
o \emph{\textcolor{red}{falso}}.

I vincoli si possono classificare in:
\begin{itemize}
    \item Vincoli \textcolor{purple}{intrarelazionali}: ovvero quelli che devono essere rispettati da valori della relazioni presa in considerazione, e possono essere:
        \begin{itemize}
            \item Vincoli sul \textcolor{purple}{dominio}, che coinvolgono un solo \emph{attributo}.
            \item Vincoli di \textcolor{purple}{ennupla}, che esprimono condizioni sui valori di ogni ennupla, indipendentemente dalle altre ennuple.
        \end{itemize}
    \item Vincoli \textcolor{purple}{interrelazionali}: sono quei vincoli che devono essere rispettati da valori presenti in relazioni diverse.
\end{itemize}

\subsection{Chiavi}

\begin{definition}[Superchiave]
    Un insieme $K$ di attributi viene chiamato \textbf{\textcolor{purple}{superchiave}} per una
    relazione $r$, se $r$ non contiene due ennuple distinte $t_1, t_2$ tali che $t_1[K] = t_2[K]$.
\end{definition}

\begin{definition}[Chiave]
    $K$ invece viene definito \textbf{\textcolor{purple}{chiave}} per $r$ se è una \emph{\textcolor{purple}{superchiave minimale}} per $r$, ovvero
    non deve contenere altre \emph{superchiavi}.
\end{definition}