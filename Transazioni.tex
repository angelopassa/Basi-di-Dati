\newpage
\section{Gestione delle Transazioni}

Le \textbf{\textcolor{purple}{transazioni}} rappresentano l'unità
di lavoro elementare (costituita da un insieme di istruzioni SQL) che
modificano il contenuto di un database.

Le \emph{transazioni} hanno questa sintassi:
\begin{multicols}{2}
    \begin{lstlisting}[
        language=SQL,
        showspaces=false,
        basicstyle=\ttfamily,
        numbers=left,
        numberstyle=\tiny,
        commentstyle=\color{gray}
    ]
    begin transaction
    [Istruzioni SQL]
    commit work
    \end{lstlisting}
    \columnbreak
    \begin{lstlisting}[
        language=SQL,
        showspaces=false,
        basicstyle=\ttfamily,
        numbers=left,
        numberstyle=\tiny,
        commentstyle=\color{gray}
    ]
    begin transaction
    [Istruzioni SQL]
    if [Condizione] commit work;
    else rollback work
    \end{lstlisting}
\end{multicols}

\paragraph{\textcolor{purple}{Proprietà delle Transazioni}}

\begin{itemize}
    \item \textcolor{purple}{A}tomicity: le transazioni deve essere eseguite in modo tale
        che i comandi SQL all'interno di esse vengano o eseguiti tutti o non eseguiti proprio.

        Quindi le transazioni che terminano prematuramente sono trattate dal sistema come se non fossaro
        mai iniziate, quindi eventuali loro effetti sul database vengono annullati.
    \item \textcolor{purple}{C}onsistency: le transazioni devono lasciare il database in uno stato
        \emph{consistente}, ovvero in uno stato dove i vincoli di integrità non devono essere violati.
    \item \textcolor{purple}{I}solation: l'esecuzione di una transazione deve essere indipendente dalle altre.

        Quindi nel caso di esecuzioni concorrenti di più transazioni, il loro effetto complessivo
        è quello di un'esecuzione seriale.
    \item \textcolor{purple}{D}urability: l'effetto di una transazione che ha eseguito un \verb|commit work|
        non deve andare perso.

        Quindi le modifiche sul database di una transazione terminata con successo
        sono permanenti, ovvero non possono essere alterate da eventuali malfunzionamenti.
\end{itemize}

La \emph{Gestione delle Transazioni} comprende la \emph{Gestione dell'Affidabilità} e
la \emph{Gestione della Concorrenza}.

\subsection{Gestione dell'Affidabilità}

\begin{definition}[Transazione]
    È un'unità logica di elaborazione che corrisponde ad una serie
    di operazioni fisiche elementari sul database.

    Una \textbf{\textcolor{purple}{transazione}} viene anche vista come una sequenza di azioni
    di lettura e scrittura in memoria permanente e di azioni di elaborazione dati iin memoria temporanea.
\end{definition}

Il \textbf{\textcolor{purple}{Gestore dell'Affidabilità}} garantisce le proprietà di
\emph{\textcolor{purple}{Atomicity}} e di \emph{\textcolor{purple}{Durability}}, è
responsabile dell'implementazione dei comandi \verb|begin transaction|, \verb|commit| e
\verb|rollback| e di ripristinare il sistema dopo malfunzionamenti software (\emph{ripresa a caldo})
e hardware (\emph{a freddo}). \\

Per aumentare l'efficienza, tutti i \emph{DBMS} dispongono di un buffer
temporaneo di dati in memoria principale, il quale viene periodicamente scritto
in memoria secondaria.

Le transazioni che interessano a noi sono solo quelle di lettura e scrittura
indicate rispettivamente con $r_i[x]$ e $w_i[x]$. Il singolo dato coinvolto può
essere un \emph{record}, un \emph{campo} di un record o una \emph{pagina}, per semplicità
si tratterà solo di pagine.

L'operazione $r_i[x]$ comporta la lettura di una pagine nel buffer se non è già presente.

L'operazione $w_i[x]$, invece, comporta le lettura nel buffer di una pagina se non è
già presente, e la sua modifica, ma non necessariamente la sua scrittura in memoria permanente.
Questo è il motivo per cui in caso di malfunzionamenti si potrebbe perdere l'effetto
di un operazione.

\paragraph{\textcolor{purple}{Tipi di Malfunzionamento}}
\begin{itemize}
    \item Fallimenti di \textcolor{purple}{transazioni}: non comportano la perdita di dati
        in memoria permanente o temporanea, ma sono dovuti a violazioni di vincoli di integrità e
        di protezione.
    \item Fallimenti di \textcolor{purple}{sistema}: comportano la perdita di dati in memoria
        temporanea ma non permanente.
    \item \textcolor{purple}{Disastri}: comportano perdita di dati in memoria permanente, può
        essere dovuto per esempio al danneggiamento della periferica.
\end{itemize}