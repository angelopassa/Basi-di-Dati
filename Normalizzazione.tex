\newpage

\section{Normalizzazione}

La \emph{teoria della progettazione relazione} studia le anomalie all'interno
degli schemi relazionali e li elimina tramite un processo di \textbf{\textcolor{purple}{normalizzazione}}.

Le \emph{anomalie} possono essere delle ridondanze o delle potenziali
inconsistenze quando si effettuano modifiche (anche inserimenti e cancellazioni) nelle istanze.

\begin{definition}[Forma Normale]
    Una \textbf{\textcolor{purple}{forma normale}} è una proprietà di un
    database che ne garantisce l'assenza di anomalie.
\end{definition}

Per seguire una corretta progettazione si fà in modo che non si uniscano in un'unica
relazione attributi che provengano da più classi e associazioni; si cerca di ridurre
al minimo la ridondanza; si evita di avere attributi che abbiano frequentemente
valori nulli; ed infine bisogna fare in modo che le relazioni possano essere riunite
tramite \verb|JOIN| con condizioni di uguaglianze e su attributi che siano o \emph{chiavi primarie}
o \emph{chiavi esterne} in modo tale da evitare la generazione di \emph{\textcolor{purple}{tuple spurie}}.

\begin{definition}[Istanza Valida]
    Un'\textbf{\textcolor{purple}{istanza valida}} su una relazione è un'istanza a cui
    viene attribuita anche una nozione semantica, e che quindi è semanticamente corretta
    nel dominio del discorso.
\end{definition}

\begin{definition}[Dipendenza Funzionale]
    Dato uno schema $R(T)$ e due sottoinsiemi di attributi $X, Y \subseteq T$, una \textbf{\textcolor{purple}{dipendenza funzionale}}
    fra gli attributi di $X$ e $Y$ è un vincolo espresso nella forma $X \rightarrow Y$, e sta a significare
    che gli attributi di $X$ determinano funzionalmente quelli di $Y$ se per ogni \emph{istanza valida} $r$ su $R(T)$ vale che:
    \begin{equation*}
        \forall \; t_1, t_2 \in r\;.\;t_1[X] = t_2[X] \Rightarrow t_1[Y] = t_2[Y]
    \end{equation*}
\end{definition}