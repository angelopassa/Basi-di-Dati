\newpage

\section{Normalizzazione}

La \emph{teoria della progettazione relazionale} studia le anomalie all'interno
degli schemi relazionali e li elimina tramite un processo di \textbf{\textcolor{purple}{normalizzazione}}.

Le \emph{anomalie} possono essere delle ridondanze o delle potenziali
inconsistenze quando si effettuano modifiche (anche inserimenti e cancellazioni) nelle istanze.

\begin{definition}[Forma Normale]
    Una \textbf{\textcolor{purple}{forma normale}} è una proprietà di un
    database che ne garantisce l'assenza di anomalie.
\end{definition}

Per seguire una corretta progettazione si fà in modo che non si uniscano in un'unica
relazione attributi che provengano da più classi e associazioni; si cerca di ridurre
al minimo la ridondanza; si evita di avere attributi che abbiano frequentemente
valori nulli; ed infine bisogna fare in modo che le relazioni possano essere riunite
tramite \verb|JOIN| con condizioni di uguaglianze e su attributi che siano o \emph{chiavi primarie}
o \emph{chiavi esterne} in modo tale da evitare la generazione di \emph{\textcolor{purple}{tuple spurie}}.

\begin{definition}[Istanza Valida]
    Un'\textbf{\textcolor{purple}{istanza valida}} su una relazione è un'istanza a cui
    viene attribuita anche una nozione semantica, e che quindi è semanticamente corretta
    nel dominio del discorso.
\end{definition}

\begin{definition}[Dipendenza Funzionale]
    Dato uno schema $R(T)$ e due sottoinsiemi di attributi $X, Y \subseteq T$, una \textbf{\textcolor{purple}{dipendenza funzionale}}
    fra gli attributi di $X$ e $Y$ è un vincolo espresso nella forma $X \rightarrow Y$, e sta a significare
    che gli attributi di $X$ determinano funzionalmente quelli di $Y$ se per ogni \emph{istanza valida} $r$ su $R(T)$ vale che:
    \begin{equation*}
        \forall \; t_1, t_2 \in r\;.\;t_1[X] = t_2[X] \Rightarrow t_1[Y] = t_2[Y]
    \end{equation*}
\end{definition}

Quando un'istanza $r_0$ soddisfa una \textcolor{purple}{dipendenza funzionale} $X \rightarrow Y$ si denota
con $r_0 \models X \rightarrow Y$.

\begin{definition}[Dipendenza Funzionale Atomica]
    Ogni \textcolor{purple}{dipendenza funzionale} $X \rightarrow A_1, A_2, \dots, A_n$
    si può scomporre nelle \textbf{\textcolor{purple}{dipendenze funzionali atomiche}}
    $X \rightarrow A_1$, $X \rightarrow A_2$, \dots, $X \rightarrow A_n$.
\end{definition}

\begin{definition}[Dipendenza Funzionale Non Banale]
    Una \textcolor{purple}{dipendenza funzionale} del tipo
    $X \rightarrow A$ si dice \emph{\textbf{\textcolor{purple}{non banale}}}
    se $A \not\subseteq X$.
\end{definition}

Le dipendenze funzionali possono essere espresse per:
\begin{itemize}
    \item \textcolor{purple}{Espressione Diretta} ($P \Rightarrow Q$): $X_= \Rightarrow Y_=$, cioè
        se in due righe i valori degli attributi in $X$ sono uguali, allora devono esserlo anche in $Y$.
    \item \textcolor{purple}{Contrapposizione} ($\neg Q \Rightarrow \neg P$): $Y_{\neq} \Rightarrow X_{\neq}$, cioè
        se in due righe i valori degli attributi in $Y$ sono diversi, allora devono esserlo anche in $X$.
    \item \textcolor{purple}{Per Assurdo}: $\neg(X_= \land Y_{\neq})$ o $X_= \land Y_{\neq} \Rightarrow False$, ovvero non
        ci possono essere due righe con i valori degli attributi in $X$ uguali e in $Y$ diversi.
\end{itemize}

Questi sono 3 modi per esprimere la stessa dipendenza funzionale.

La notazione $R<T, F>$ indica una relazione $R$ definita su uno schema
di attributi $T$ e dipendenze funzionali $F$.

\begin{definition}[Dipendenza Funzionale Completa]
    Una \textcolor{purple}{dipendenza funzionale} $X \rightarrow Y$ si dice \textbf{\emph{\textcolor{purple}{completa}}}
    quando per ogni $W \subset X$ non vale $W \rightarrow Y$.
\end{definition}

Se $X$ è una \emph{superchiave} allora $X$ determina ogni altro attributo
della relazione, quindi $X \rightarrow T$.

Se $X$ è una \emph{chiave} allora $X \rightarrow T$ è una \emph{dipendenza funzionale completa}.

\begin{definition}[Dipendenza Implicata]
    Sia $F$ un insieme di dipendenze funzionali sullo schema $R(T)$,
    allora $F$ \emph{\textcolor{purple}{implica logicamente}} $X \rightarrow Y$ ($F \models X \rightarrow Y$),
    se ogni istanza di relazione $r$ su $R(T)$ che soddisfa tutte le dipendenze funzionali
    in $F$, soddisfa anche $X \rightarrow Y$.
\end{definition}

\paragraph{\textcolor{purple}{Regole di Inferenza} o \textcolor{purple}{Assiomi di Armstrong}}
\begin{itemize}
    \item \textcolor{purple}{Riflessività}: $Y \subseteq X \Rightarrow X \rightarrow Y$.
    \item \textcolor{purple}{Arricchimento}: $X \rightarrow Y \land Z \subseteq T \Rightarrow XZ \rightarrow YZ$.
    \item \textcolor{purple}{Transitività}: $X \rightarrow Y \land Y \rightarrow Z \Rightarrow X \rightarrow Z$.
\end{itemize}

\begin{definition}[Derivazione]
    Dato $F$ un insieme di \emph{dipendenze funzionali}, si dice
    che $X \rightarrow Y$ è \emph{\textcolor{purple}{derivabile}} da
    $F$ (con la notazione $F \vdash X \rightarrow Y$) se $X \rightarrow Y$ può
    essere inferito da $F$ utilizzando gli \emph{Assiomi di Armstrong}.

    Una \textbf{\textcolor{purple}{derivazione}} della dipendenza funzionale $f$ a partire
    da $F$ è una sequenza finita di dipendenze $f_1, \dots, f_m$, dove $f_m = f$ ed
    ogni $f_i$ è o un elemento di $F$ oppure è ottenuta dalle precedenti
    $f_1, \dots, f_{n-1}$ dipendenze della derivazione applicando ad una di esse una regola di inferenza.
\end{definition}

Usando la \emph{\textcolor{purple}{derivazione}} si dimostrano le seguenti regole:
\begin{itemize}
    \item \textcolor{purple}{Unione}: $\{X \rightarrow Y, X \rightarrow Z\} \vdash X \rightarrow YZ$.
    \item \textcolor{purple}{Decomposizione}: Dato $Z \subseteq Y$, allora $\{X \rightarrow Y\} \vdash X \rightarrow Z$.
\end{itemize}

\begin{theorem}[Completezza e Correttezza degli Assiomi di Armstrong]
    Gli \emph{Assiomi di Armstrong} sono \emph{\textcolor{purple}{corretti}}
    e \emph{\textcolor{purple}{completi}}. Ovvero vale la seguente equivalenza
    $\models\;\;\equiv\;\;\vdash$:
    \begin{itemize}
        \item \textcolor{purple}{Correttezza}: $\forall\;f,\;F \vdash f \Rightarrow F \models f$.
        \item \textcolor{purple}{Completezza}: $\forall\;f,\;F \models f \Rightarrow F \vdash f$.
    \end{itemize}
\end{theorem}

\begin{definition}[Chiusura di $F$]
    Dato un insieme $F$ di dipendenze funzionali, la \textbf{\textcolor{purple}{chiusura}} di
    $F$ denotata con \textbf{\textcolor{purple}{$F^+$}} è:
    \begin{equation*}
        F^+ = \{X \rightarrow Y \;|\; F \vdash X \rightarrow Y\}
    \end{equation*}
\end{definition}

\paragraph{\textcolor{purple}{Problema dell'Implicazione}} Consiste nel decidere se una \emph{dipendenza funzionale}
$X \rightarrow Y$ appartiene o no ad $F^+$. Applicando un algoritmo banale, si genera $F^+$ applicando ad $F$
gli \emph{Assiomi di Armstrong}, ma questo ha ovviamente un ordine di complessità esponenziale.

\begin{definition}[Chiusura di $X$ rispetto ad $F$]
    Dato $R<T, F>$ e un sottoinsiemi di attributi $X \subseteq T$, la
    \textbf{\textcolor{purple}{chiusura di $X$ rispetto ad $F$}} denotata con
    \textbf{\textcolor{purple}{$X_{F}^{+}$}} è l'insieme:
    \begin{equation*}
        X_{F}^{+} = \{A_i \in T \;|\; F \vdash X \rightarrow A_i\}
    \end{equation*}
\end{definition}

\begin{theorem}
    $F \vdash X \rightarrow Y \Leftrightarrow Y \subseteq X_{F}^{+}$
\end{theorem}

\paragraph{\textcolor{purple}{Algoritmo per Calcolare $X_{F}^+$}}
Si parte da un insieme di attributi $X$ e da un insieme di dipendenze $F$:
\begin{enumerate}
    \item Si inizializza $X^+$ con l'insieme $X$, dato che per la \emph{riflessività}
        vale che $X \rightarrow X$ indipendentemente da $F$.
    \item Poi, se fra le dipendenze di $F$ c'è una dipendenza $Y \rightarrow A$ con
        $Y \subseteq X^+$ allora si inserisce $A$ in $X^+$ ($X^+ = X^+ \cup {A}$). Questo
        vale dato che se se $Y \subseteq X^+$ allora $Y$ si può derivare funzionalmente da $F$ e
        per \emph{transitività} si arriva a $Y \rightarrow A$.
    \item Si ripete il secondo passo finchè non ci sono altri attributi da aggiungere a $X^+$, quindi abbiamo
        la certezza che l'algoritmo termini dato che la cardinalità di $X$ è finita.
    \item Alla fine $X_{F}^+ = X^+$.
\end{enumerate}

\begin{definition}[Chiave Candidata]
    Dato una relazione definita su $R<T, F>$ si dice che un insieme
    di attributi $W \subseteq T$ è una \textbf{\textcolor{purple}{chiave candidata}}
    della relazione se:
    \begin{itemize}
        \item $W \rightarrow T \in F^+$, ovvero $W$ è una \emph{\textcolor{purple}{superchiave}}.
        \item $\forall \; V \subset W, \; V \rightarrow T \not\in F^+$, ovvero se $V$ è contenuto
            in $W$ allora $V$ non deve essere una superchiave.
    \end{itemize}
\end{definition}

\begin{definition}[Attributo Primo]
    Un \textbf{\textcolor{purple}{attributo primo}} è un attributo che appartiene
    ad almeno una chiave.
\end{definition}

\begin{definition}[Copertura]
    Due insiemi di \emph{dipendenze funzionali} sono \textbf{\textcolor{purple}{equivalenti}}
    ovvero $F \equiv G$, se e solo se $F^+ = G^+$. Si dice anche che $F$ è
    una \textbf{\textcolor{purple}{copertura}} di $G$ e viceversa.
\end{definition}

\begin{definition}[Attributo Estraneo]
    Dato un insieme di dipendenze funzionali $F$ una dipendenza funzionale
    $X \rightarrow Y \in F$, si dice che $X$ contiene un \textbf{\textcolor{purple}{attributo estraneo}}
    $A_i$ se e solo se:
    \begin{equation*}
        (X - \{A_i\} \rightarrow Y \in F^+)
    \end{equation*}
    Ovvero $F \vdash (X - \{A_i\}) \rightarrow Y$. Per verificare che $A_i$ è estraneo
    basta calcolare $(X - \{A_i\})^+$ e verificare che includa $Y$.

    In questo caso l'insieme $F$ può essere riscritto eliminando $X \rightarrow Y$ e sostituendolo
    con $(X - \{A_i\}) \rightarrow Y$.
\end{definition}

\begin{definition}[Dipendenza Ridondante]
    Dato un insieme di dipendenze funzionali $F$, si dice $X \rightarrow Y$ è una \textbf{\textcolor{purple}{dipendenza ridondante}}
    se e solo se:
    \begin{equation*}
        (F - \{X \rightarrow Y\})^+ = F^+
    \end{equation*}
    che equivale a dire che $F - \{X \rightarrow Y\} \vdash X \rightarrow Y$. Per verificare che $X \rightarrow A$ è
    ridondante, la si elimina da $F$ e si calcola $X^+$, se include $A$ allora la dipendenza è ridondante.
\end{definition}

\paragraph{\textcolor{purple}{Algoritmo Per Trovare Le Chiavi}} Ci si basa su due proprietà, se un attributo $A$ non
appare a destra di nessuna dipendenza funzionale $F$ allora $A$ appartiene
ad ogni chiave della relazione; invece se un attributo $B$ appare solo a destra in tutte
le dipendenze funzionali in cui è coinvolto, allora $B$ non appartiene ad alcune chiave.

Detto ciò, si parte dall'insieme di attributi che è presente solo a sinistra delle dipendenze
e si calcola la chiusura, se è chiave si termina, altrimenti si aggiunge iterativamente un nuovo attributo
all'insieme e si prova a calcolare la chiusura (ovviamente non un attributo che appare solo a destra delle dipendenze).

\begin{definition}[Copertura Canonica]
    Un insieme di dipendenze funzionali $F$ è detto una \textbf{\textcolor{purple}{copertura canonica}} se e solo se:
    \begin{itemize}
        \item La parte destra di ogni dipendenza funzionale in $F$ è un singolo attributo.
        \item Non esistono \emph{attributi estranei}.
        \item Non esistono \emph{dipendenze ridondanti}.
    \end{itemize}
\end{definition}

\begin{theorem}
    Per ogni insieme $F$ esiste una sua \emph{copertura canonica}.
\end{theorem}

\paragraph{\textcolor{purple}{Ridondanze}} Le \emph{\textcolor{purple}{ridondanze}} possono essere di due tipi:
\begin{itemize}
    \item \textcolor{purple}{Concettuale}: nel database sono memorizzate informazioni che possono essere ricavate
        da altre già presenti.
    \item \textcolor{purple}{Logica}: esistono duplicazioni sui dati che possono generare anomalie, come visto fin qui.
\end{itemize}

\subsection{Decomposizione di Schemi}

In linea generale, per eliminare delle \emph{anomalie} da uno schema occorre decomporlo in schemi più piccoli ma che
siano equivalenti. L'intuizione è quella di creare uno schema per ogni insieme di dipendenze funzionali che hanno la parte
a sinistra uguale e che rappresenti una chiave. Ma questa è una soluzione che non può funzionare sempre.

\begin{definition}[Decomposizione]
    Data una relazione definita su $R(T)$, $\rho = \{R_1(T_1), \dots, R_k(T_k)\}$ è una
    \textbf{\textcolor{purple}{decomposizione}} di $R$ se e solo se $\cup \; T_i = T$.
    Due proprietà che è desiderabile avere in una \emph{decomposizione} sono:
    \begin{itemize}
        \item La \emph{\textcolor{purple}{conservazione dei dati}}, intesa come \emph{nozione semantica}.
        \item E la \emph{\textcolor{purple}{conservazione delle dipendenze}}.
    \end{itemize}
\end{definition}

\begin{definition}[Decomposizione che Preserva i Dati]
    Data una \emph{decomposizione} $\rho = \{R_1(T_1), \dots, R_k(T_k)\}$ su $R(T)$, si dice
    che \emph{\textcolor{purple}{preserva i dati}} se e solo se per ogni \emph{istanza valida} $r$ su $R$:
    \begin{equation*}
        r = (\pi_{T_1}(r)) \vee (\pi_{T_2}(r)) \vee \dots \vee (\pi_{T_k}(r))
    \end{equation*}
    dove $\vee$ è l'operatore di \emph{natural join}. È importante sottolineare la presenza
    del simbolo $=$ nell'espressione e non $\subseteq$, dato che non vogliamo la produzione di \emph{ennuple spurie}.
\end{definition}

\begin{theorem}[Decomposizioni Binarie]
    Sia $R<T, F>$ uno schema di relazione, la \emph{decomposizione}
    $\rho = \{R_1(T_1), R_1(T_2)\}$ \emph{\textcolor{purple}{preserva i dati}}
    se e solo se:
    \begin{equation*}
        (T_1 \cap T_2 \rightarrow T_1) \in F^+
    \end{equation*}
    oppure
    \begin{equation*}
        (T_1 \cap T_2 \rightarrow T_2) \in F^+
    \end{equation*}
    Ciò significa che l'insieme degli attributi comuni alle due relazioni deve essere chiave
    per almeno una delle due.
\end{theorem}

\begin{proof}[Dimostrazione]
    Si prende un'istanza di relazione $r$ sugli attributi $ABC$ e si considerano
    le sue proiezioni $r_1$ su $AB$ e $r_2$ su $AC$.
    Si suppone che $A$ è chiave per $r$ su $AC$ (la dimostrazione è simmetrica per $AB$) e
    che quindi $r$ soddisfi la dipendenza funzionale $A \rightarrow C$. Quindi sulla
    proiezione $AC$ non ci sono tuple diverse quando in $A$ ci sono valori uguali.
    Se $t = (a,b,c)$ è una tupla del \verb|JOIN| tra $r_1$ e $r_2$ si mostra che appartiene
    sicuramente ad $r$: $t$ sarà ottenuta dal \verb|JOIN| tra $t_1 = (a,b)$ di $r_1$
    e $t_2 = (a,c)$ di $r_2$. Quindi per definizione di proiezione in $r$ esisteranno al più due tuple
    $t_{1}^{'} = (a,b,*)$ e $t_{2}^{'} = (a,*,c)$, dove $*$ rappresenta un valore non noto.
    Poichè $A \rightarrow C$, allora esiste un unico valore di $C$ associato allo stesso valore di $A$;
    dato che $t_1$ e $t_{1}^{'}$ hanno lo stesso valore di $A$, il valore non noto $*$ in $t_{1}^{'}$
    deve essere obbligatoriamente $c$. Quindi $t = t_{1}^{'}$.
\end{proof}

Anche se una decomposizione è senza perdita può comunque presentare problemi di \emph{\textcolor{purple}{conservazione delle dipendenze}}.
Si dice che una \emph{decomposizione conserva le dipendenze} se ciascuna delle dipendenze funzionali dello schema originario
coinvolge attributi che compaiono tutti insieme in uno degli schemi decomposti.
Altrimenti, una soluzione è di utilizzare query $SQL$ di verifica (ad esempio facendo uso di \emph{trigger}) per
far in modo che le dipendenze vengano soddisfatte lo stesso.

\begin{definition}[Proiezione]
    Dato lo schema $R<T,F>$ e un sottoinsieme di attributi $T_1 \subseteq T$,
    la \textbf{\textcolor{purple}{proiezione}} di $F$ su $T_1$ è:
    \begin{equation*}
        \pi_{T_1}(F) = \{X \rightarrow Y \in F^+ \;|\; XY \subseteq T_1\}
    \end{equation*}
\end{definition}

\paragraph{\textcolor{purple}{Algoritmo per Calcolare $\pi_{T_1}(F)$}}
L'algoritmo è il seguente:
\begin{algorithm}
    \begin{algorithmic}
        \ForAll{$Y \subseteq T_1$}
            \State $Z \gets Y^+$
            \State \textbf{output} $Y \rightarrow Z \cap T_1$
        \EndFor
    \end{algorithmic}
\end{algorithm}

\begin{definition}[Decomposizione che Preserva le Dipendenze]
    Dato uno schema $R<T,F>$, la decomposizione $\rho = \{R_1, \dots, R_n\}$
    \emph{\textcolor{purple}{preserva le dipendenze}} se e solo se l'unione delle
    dipendenze in $\pi_{T_i}(F)$ è una \textbf{\textcolor{purple}{copertura}} di $F$.

    Il problema di stabilire se una decomposizione preserva o no le dipendenze ha tempo polinomiale.
\end{definition}

\begin{theorem}
    Sia $\rho = {R_i<T_i, F_i>}$ una decomposizione di $R<T, F>$ che preserva
    le dipendenze e tale che per qualche $j$, $T_j$ è una superchiave per $R<T,F>$.

    Allora $\rho$ preserva anche i \textbf{\textcolor{purple}{dati}}.
\end{theorem}

Una \textcolor{purple}{decomposizione} dovrebbe sempre essere \emph{\textcolor{purple}{senza perdita}},
quindi in grado di ricostruire tutte le tuple originarie senza generare tuple spurie; e deve \emph{\textcolor{purple}{conservare le dipendenze}},
in modo da garantire i vincoli di integrità (rappresentate dalle dipendenze funzionali).

\subsection{Forme Normali}

\begin{definition}[Forma Normale]
    Una \textbf{\textcolor{purple}{forma normale}} è una proprietà di un database relazionale che
    ne garantisce l'assenza di \emph{ridondanze} e di comportamenti non deriderabili in fase di aggiornamento.
    
    La prima forma normale \textbf{\textcolor{purple}{1FN}} impone restrizioni sul tipo degli attributi, ovvero ogni attributo
    deve avere un tipo \emph{\textcolor{purple}{elementare}}. Mentre
    la \textbf{\textcolor{purple}{2FN}}, \textbf{\textcolor{purple}{3FN}} e la
    \textbf{\textcolor{purple}{FNBC}} (\emph{Boyce-Codd}, che è la più restrittiva) impongono
    restrizioni sulle dipendenze.
\end{definition}

\subsubsection{BCNF - Forma Normale di Boyce e Cood}

\begin{definition}[BCNF]
    Un'istanza di relazione $r$ definita su $R<T,F>$ è in \textbf{\textcolor{purple}{BCNF}}
    se e solo se per ogni dipendenza funzionale \emph{non banale} $X \rightarrow Y \in F^+$, $X$
    contiene una \textbf{chiave} $K$ di $r$, ovvero se $X$ è una \emph{superchiave}.

    Infatti se esistesse una dipendenza $X \rightarrow Y$ non banale in cui
    $X$ non è superchiave, vuol dire che $X$ modella un'entità diversa
    da quella modellata da $R$, e che quindi è possibile decomporre $R$.
\end{definition}

\paragraph{\textcolor{purple}{Algoritmo di Analisi}} Questo algoritmo riceve in
input uno schema $R<T,F>$ con $F$ che è una \emph{copertura canonica} e dà in output
una \emph{decomposizione in BCNF} che \emph{preserva i dati} ma non necessariamente le
dipendenze. L'algoritmo è il seguente e la sua complessità è esponenziale:

\begin{algorithm}
    \begin{algorithmic}
        \State $\rho \gets \{R<T,F>\}$
        \While{Esiste in $\rho$ una $R_i<T_i,F_i>$ non in BCNF per la DF $X \rightarrow Y$}
            \State $T_a \gets X^+$
            \State $F_a \gets \pi_{T_a}(F_i)$
            \State $T_b \gets T_i - X^+ + X$
            \State $F_b \gets \pi_{T_b}(F_i)$
            \State $\rho \gets \rho - R_i + \{R_a<T_a,F_a>, R_b<T_b,F_b>\}$
        \EndWhile
    \end{algorithmic}
\end{algorithm}

\subsubsection{3NF}

\begin{definition}[3NF]
    Un'istanza di relazione $r$ definita su $R<T,F>$ è in \textbf{\textcolor{purple}{3NF}}
    se e solo se per ogni dipendenza funzionale \emph{non banale} $X \rightarrow Y \in F^+$, $X$
    contiene una \textbf{chiave} $K$ di $r$, oppure ogni attributo in $Y$ è contenuto in almeno una
    \textbf{chiave} $K$ di $r$. Quindi se $X$ è una \emph{superchiave} oppure
    se $Y$ è \emph{primo}.
\end{definition}

La \textbf{3FN} è meno restrittiva della \textbf{FNBC}, dato che tollera alcune ridondanze ed anomaile sui dati,
quindi certifica uno schema di qualità inferiore. Il vantaggio, però, è che la \textbf{3FN} è sempre
ottenibile qualunque sia lo schema di partenza.

\paragraph{\textcolor{purple}{Algoritmo di Sintesi}} L'algoritmo prende in
input un insieme $T$ di attributi e un insieme $F$ di dipendenze su $T$. Come
output dà una decomposizione $\rho = \{S_i\}_{i = 1, \dots, n}$ di $T$ tale
che vengano preservati dati e dipendenze e che ogni $S_i$ sia in \textbf{3FN} rispetto
alle proiezioni di $F$ su $S_i$. L'algoritmo è il seguente:
\begin{enumerate}
    \item Si trova una \emph{copertura canonica} $G$ di $F$ e si pone
        $\rho = \{\}$.
    \item Si sostituisce in $G$ ogni insieme di dipendenze con lo stesso determinante
        $X$ ($X \rightarrow A_1, \dots, X \rightarrow A_h$) con la dipendenza $X \rightarrow A_1, \dots, A_h$.
    \item Per ogni dipendenza $X \rightarrow Y$ in $G$ si aggiunge in $\rho$ uno schema
        con attributi $XY$. $X$ diventa la \emph{chiave} della relazione.
    \item Si elimina ogni schema di $\rho$ che è contenuto in un qualche altro schema.
    \item Se la decomposizione non contiene neanche uno schema i cui attributi costituiscono
        una superchiave per $R$, allora si aggiunge a $\rho$ uno schema con attributi $W$,
        dove $W$ è una chiave di $R$.
\end{enumerate}