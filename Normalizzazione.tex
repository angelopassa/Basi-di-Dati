\newpage

\section{Normalizzazione}

La \emph{teoria della progettazione relazione} studia le anomalie all'interno
degli schemi relazionali e li elimina tramite un processo di \textbf{\textcolor{purple}{normalizzazione}}.

Le \emph{anomalie} possono essere delle ridondanze o delle potenziali
inconsistenze quando si effettuano modifiche (anche inserimenti e cancellazioni) nelle istanze.

\begin{definition}[Forma Normale]
    Una \textbf{\textcolor{purple}{forma normale}} è una proprietà di un
    database che ne garantisce l'assenza di anomalie.
\end{definition}

Per seguire una corretta progettazione si fà in modo che non si uniscano in un'unica
relazione attributi che provengano da più classi e associazioni; si cerca di ridurre
al minimo la ridondanza; si evita di avere attributi che abbiano frequentemente
valori nulli; ed infine bisogna fare in modo che le relazioni possano essere riunite
tramite \verb|JOIN| con condizioni di uguaglianze e su attributi che siano o \emph{chiavi primarie}
o \emph{chiavi esterne} in modo tale da evitare la generazione di \emph{\textcolor{purple}{tuple spurie}}.

\begin{definition}[Istanza Valida]
    Un'\textbf{\textcolor{purple}{istanza valida}} su una relazione è un'istanza a cui
    viene attribuita anche una nozione semantica, e che quindi è semanticamente corretta
    nel dominio del discorso.
\end{definition}

\begin{definition}[Dipendenza Funzionale]
    Dato uno schema $R(T)$ e due sottoinsiemi di attributi $X, Y \subseteq T$, una \textbf{\textcolor{purple}{dipendenza funzionale}}
    fra gli attributi di $X$ e $Y$ è un vincolo espresso nella forma $X \rightarrow Y$, e sta a significare
    che gli attributi di $X$ determinano funzionalmente quelli di $Y$ se per ogni \emph{istanza valida} $r$ su $R(T)$ vale che:
    \begin{equation*}
        \forall \; t_1, t_2 \in r\;.\;t_1[X] = t_2[X] \Rightarrow t_1[Y] = t_2[Y]
    \end{equation*}
\end{definition}

Quando un'istanza $r_0$ soddisfa una \textcolor{purple}{dipendenza funzionale} $X \rightarrow Y$ si denota
con $r_0 \models X \rightarrow Y$.

\begin{definition}[Dipendenza Funzionale Atomica]
    Ogni \textcolor{purple}{dipendenza funzionale} $X \rightarrow A_1, A_2, \dots, A_n$
    si può scomporre nelle \textbf{\textcolor{purple}{dipendenze funzionali atomiche}}
    $X \rightarrow A_1$, $X \rightarrow A_2$, \dots, $X \rightarrow A_n$.
\end{definition}

\begin{definition}[Dipendenza Funzionale Non Banale]
    Una \textcolor{purple}{dipendenza funzionale} del tipo
    $X \rightarrow A$ si dice \emph{\textbf{\textcolor{purple}{non banale}}}
    se $A \not\subseteq X$.
\end{definition}

Le dipendenze funzionali possono essere espresse per:
\begin{itemize}
    \item \textcolor{purple}{Espressione Diretta} ($P \Rightarrow Q$): $X_= \Rightarrow Y_=$, cioè
        se in due righe i valori degli attributi in $X$ sono uguali, allora devono esserlo anche in $Y$.
    \item \textcolor{purple}{Contrapposizione} ($\neg Q \Rightarrow \neg P$): $Y_{\neq} \Rightarrow X_{\neq}$, cioè
        se in due righe i valori degli attributi in $Y$ sono diversi, allora devono esserlo anche in $X$.
    \item \textcolor{purple}{Per Assurdo}: $\neg(X_= \land Y_{\neq})$ o $X_= \land Y_{\neq} \Rightarrow False$, ovvero non
        ci possono essere due righe con i valori degli attributi in $X$ uguali e in $Y$ diversi.
\end{itemize}

Questi sono 3 modi per esprimere la stessa dipendenza funzionale.

La notazione $R<T, F>$ indica una relazione $R$ definita su uno schema
di attributi $T$ e dipendenze funzionali $F$.

\begin{definition}[Dipendenza Funzionale Completa]
    Una \textcolor{purple}{dipendenza funzionale} $X \rightarrow Y$ si dice \textbf{\emph{\textcolor{purple}{completa}}}
    quando per ogni $W \subset X$ non vale $W \rightarrow Y$.
\end{definition}

Se $X$ è una \emph{superchiave} allora $X$ determina ogni altro attributo
della relazione, quindi $X \rightarrow T$.

Se $X$ è una \emph{chiave} allora $X \rightarrow T$ è una \emph{dipendenza funzionale completa}.

\begin{definition}[Dipendenza Implicata]
    Sia $F$ un insieme di dipendenze funzionali sullo schema $R(T)$,
    allora $F$ \emph{\textcolor{purple}{implica logicamente}} $X \rightarrow Y$ ($F \models X \rightarrow Y$),
    se ogni istanza di relazione $r$ su $R(T)$ che soddisfa tutte le dipendenze funzionali
    in $F$, soddisfa anche $X \rightarrow Y$.
\end{definition}

\paragraph{\textcolor{purple}{Regole di Inferenza} o \textcolor{purple}{Assiomi di Armstrong}}
\begin{itemize}
    \item \textcolor{purple}{Riflessività}: $Y \subseteq X \Rightarrow X \rightarrow Y$.
    \item \textcolor{purple}{Arricchimento}: $X \rightarrow Y \land Z \subseteq T \Rightarrow XZ \rightarrow YZ$.
    \item \textcolor{purple}{Transitività}: $X \rightarrow Y \land Y \rightarrow Z \Rightarrow X \rightarrow Z$.
\end{itemize}

\begin{definition}[Derivazione]
    Dato $F$ un insieme di \emph{dipendenze funzionali}, si dice
    che $X \rightarrow Y$ è \emph{\textcolor{purple}{derivabile}} da
    $F$ (con la notazione $F \vdash X \rightarrow Y$) se $X \rightarrow Y$ può
    essere inferito da $F$ utilizzando gli \emph{Assiomi di Armstrong}.

    Una \textbf{\textcolor{purple}{derivazione}} della dipendenza funzionale $f$ a partire
    da $F$ è una sequenza finita di dipendenze $f_1, \dots, f_m$, dove $f_m = f$ ed
    ogni $f_i$ è o un elemento di $F$ oppure è ottenuta dalle precedenti
    $f_1, \dots, f_{n-1}$ dipendenze della derivazione applicando ad una di esse una regola di inferenza.
\end{definition}

Usando la \emph{\textcolor{purple}{derivazione}} si dimostrano le seguenti regole:
\begin{itemize}
    \item \textcolor{purple}{Unione}: $\{X \rightarrow Y, X \rightarrow Z\} \vdash X \rightarrow YZ$.
    \item \textcolor{purple}{Decomposizione}: Dato $Z \subseteq Y$, allora $\{X \rightarrow Y\} \vdash X \rightarrow Z$.
\end{itemize}

\begin{theorem}[Completezza e Correttezza degli Assiomi di Armstrong]
    Gli \emph{Assiomi di Armstrong} sono \emph{\textcolor{purple}{corretti}}
    e \emph{\textcolor{purple}{completi}}. Ovvero vale la seguente equivalenza
    $\models\;\;\equiv\;\;\vdash$:
    \begin{itemize}
        \item \textcolor{purple}{Correttezza}: $\forall\;f,\;F \vdash f \Rightarrow F \models f$.
        \item \textcolor{purple}{Completezza}: $\forall\;f,\;F \models f \Rightarrow F \vdash f$.
    \end{itemize}
\end{theorem}

\begin{definition}[Chiusura di $F$]
    Dato un insieme $F$ di dipendenze funzionali, la \textbf{\textcolor{purple}{chiusura}} di
    $F$ denotata con \textbf{\textcolor{purple}{$F^+$}} è:
    \begin{equation*}
        F^+ = \{X \rightarrow Y \;|\; F \vdash X \rightarrow Y\}
    \end{equation*}
\end{definition}

\paragraph{\textcolor{purple}{Problema dell'Implicazione}} Consiste nel decidere se una \emph{dipendenza funzionale}
$X \rightarrow Y$ appartiene o no ad $F^+$. Applicando un algoritmo banale, si genera $F^+$ applicando ad $F$
gli \emph{Assiomi di Armstrong}, ma questo ha ovviamente un ordine di complessità esponenziale.

\begin{definition}[Chiusura di $X$ rispetto ad $F$]
    Dato $R<T, F>$ e un sottoinsiemi di attributi $X \subseteq T$, la
    \textbf{\textcolor{purple}{chiusura di $X$ rispetto ad $F$}} denotata con
    \textbf{\textcolor{purple}{$X_{F}^{+}$}} è l'insieme:
    \begin{equation*}
        X_{F}^{+} = \{A_i \in T \;|\; F \vdash X \rightarrow A_i\}
    \end{equation*}
\end{definition}

\begin{theorem}
    $F \vdash X \rightarrow Y \Leftrightarrow Y \subseteq X_{F}^{+}$
\end{theorem}